\documentclass[a4paper]{article}
\usepackage[finnish]{babel}
\usepackage{geometry}
\usepackage[utf8]{inputenc}
\author{Juri Liukkonen, Risto Tuomainen, mitkäs teidän oikeet nimet oli?}
\title{Kassihiki-ryhmän miniprojektin loppuraportti}
\begin{document}
\maketitle
\noindent
Ryhmämme toteutti internetin kautta käytettävän viitteidenhallintajärjestelmän. Jatkuva integraatio toteutettiin Jenkingsin avulla, ja sovelluksessa käytettiin Spring-sovelluskehystä. 

Ensimmäisessä sprintissä rakennettiin sovelluskehityksessä tarvittavaa infrastruktuuria, ja toteuttiin käyttöliittymää esittelevä sivu. Olimme erityisesti alussa monien haasteiden edessä, sillä osalle ryhmästä spring oli täysin vieras teknologia. Niinpä osa ryhmän jäsenistä käytti sprintin aikana neljän tunnin työpanoksensa teknologiaan tutustumiseen, mikä vaikutti tietysti kielteisesti tuottavuuteen. Infrastruktuurin pystyttäminen ei ollut aivan helppoa muutenkaan, ja siihen kului melko paljon tunteja.  

Toisessa sprintissä tarvittu infrastruktuuri oli pystyssä alusta saakka, mutta edelleen osa ryhmästä oli käytännössä sivussa koodaamisesta lukemassa "Spring MVC - Getting Started" -opasta. Projekti ei juurikaan edennyt tämän sprintin aikana, vaan monet tehtävät siirtyivät suoraan seuraavaan sprinttiin. 

Kolmannessa sprintissä ryhmä kokoontui aiemmasta poiketin koodaamaan yhdessä. Tässäkään ei tosin yleisistä syistä johtuen saavutettu niin suurta työtehoa kuin olisi voinut toivoa, mutta joka tapauksessa organisointi sujui tällä kertaa jouhevammin. Kolmannessa sprintissä ydintoiminnallisuus saatiin toteutettua, ja demotilaisuudessa saavutettiinkin suuri voitto tämän työn tuloksena.

Yleisesti ottaen kommunikaatio ryhmän sisällä ei toiminut parhaalla mahdollisella tavalla. Ryhmä käytti irc-kanavaa keskusteluun, mutta työnjaossa oli hieman epäselvyyttä. Lisäksi kasvokkain olisi ollut helpompi opastaa niitä ryhmän jäseniä, jotka olivat aivan kujalla springin käytöstä.  


Tuomainen oppi projektin aikana paljon web-sovellusten toteuttamisesta javalla ja springillä. Lisäksi monin tavoin haasteellinen organisointi opetti paljon huonoista käytänteistä ja kirkasti selkeän kommunikaatioin tärkeyden ryhmätyön kannalta. 

Toisaalta vaikka kaikki ei ollut helppoa, lopputuloksena syntyi kuitenkin toimiva sovellus. Projekti oli muutenkin hyvin opettavainen kokemus.









\end{document}
