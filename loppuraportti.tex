\documentclass[10pt,a4paper]{article}
\usepackage[finnish]{babel}
\usepackage{geometry}
\usepackage[utf8]{inputenc}
\author{Juri Liukkonen,
	Risto Tuomainen, Timi Suominen, Esa Lindqvist}

\title{Kassihiki-ryhmän miniprojektin loppuraportti}

\begin{document}
\maketitle

\section*{}
Ryhmämme toteutti internetin kautta käytettävän
viitteidenhallintajärjestelmän. Jatkuva integraatio toteutettiin
Jenkingsin avulla, ja sovelluksessa käytettiin Spring-sovelluskehystä.

\section*{Ensimmäinen sprintti}
Ensimmäisessä sprintissä rakennettiin sovelluskehityksessä
tarvittavaa infrastruktuuria, ja toteuttiin käyttöliittymää
esittelevä sivu. Olimme erityisesti alussa monien haasteiden edessä,
sillä suurimmalle osalle ryhmästä spring oli täysin vieras teknologia.
Niinpä osa ryhmän jäsenistä käytti sprintin aikana neljän tunnin työpanoksensa
teknologiaan tutustumiseen, mikä vaikutti tietysti kielteisesti
tuottavuuteen. Infrastruktuurin pystyttäminen ei ollut aivan helppoa
muutenkaan, ja siihen kului melko paljon tunteja.

Ongelmia tuotti myös projektinhallintainfrastruktuurin pystyttäminen. Trello
ei sovellu projektinhallintatyökaluksi, mutta jo pelkästään tunnustenkin
luominen vie hirveästi aikaa.

Prosessin noudattamisessa ongelmia tuotti projektin storyjen jakaminen
mielekkään kokoisiin tehtäviin ja niiden vaatiman työpanoksen
arviointi. Tämä johtui sekä ryhmän jäsenten kokemattomuudesta yleensäkin ja
käytetyn sovelluskehyksen vieraudesta.

\section*{Toinen sprintti}
Toisessa sprintissä tarvittu infrastruktuuri oli pystyssä alusta saakka,
mutta edelleen osa ryhmästä oli käytännössä sivussa koodaamisesta
lukemassa "Spring MVC - Getting Started" -opasta. Projekti ei juurikaan
edennyt tämän sprintin aikana, vaan monet tehtävät siirtyivät
suoraan seuraavaan sprinttiin.

Koodaamisen kannalta ongelmallista oli testaaminen. Tietokanta-asioiden
testaaminen Springissä edellyttää huomattavan paljon tietämystä sen
arkkitehtuurista, mitä ryhmällä ei siinä vaiheessa ollut. (Eikä itse
asiassa ole vieläkään.) Testi-infran saattaminen kuntoon mahdollisimman
aikaisessa vaiheessa olisi saattanut helpottaa projektin etenemistä.

Prosessin noudattaminen jäi jälleen kerran kaiken muun varjoon. Tehtävien
toteutuksen vaativuus-/aika-arviot eivät pitäneet. Osasyynä voi olla
burndown chartin puuttuminen, joka olisi helpottanut tahdin seuraamista.
tekemättä, Asioiden valmiiksi
merkkaaminen projektinhallintasoftaan oli puutteellista, koska asioiden
saattaminen valmiiksi määritelmän mukaisella tavalla oli vaikeaa
testaamisen vajavaisuudesta johtuen. Myös joustavuus/agilen noudattaminen
tuotti ongelmia. 

\section*{Kolmas sprintti}
Kolmannessa sprintissä ryhmä kokoontui aiemmasta poiketin koodaamaan
yhdessä. Tässäkään ei tosin yleisistä syistä johtuen saavutettu
niin suurta työtehoa kuin olisi voinut toivoa, mutta joka tapauksessa
organisointi sujui tällä kertaa jouhevammin. Kolmannessa sprintissä
ydintoiminnallisuus saatiin toteutettua, ja demotilaisuudessa
saavutettiinkin suuri voitto tämän työn tuloksena.

Yleisesti ottaen kommunikaatio ryhmän sisällä ei toiminut parhaalla
mahdollisella tavalla. Ryhmä käytti irc-kanavaa keskusteluun, mutta
työnjaossa oli hieman epäselvyyttä. Lisäksi kasvokkain olisi ollut
helpompi opastaa niitä ryhmän jäseniä, jotka olivat aivan kujalla
Springin käytöstä.

Prosessin noudattamisesta luovuttiin kolmannella viikolla ihan täysin.

Tuomainen oppi projektin aikana paljon web-sovellusten toteuttamisesta
javalla ja springillä. Lisäksi monin tavoin haasteellinen organisointi
opetti paljon huonoista käytänteistä ja kirkasti selkeän
kommunikaatioin tärkeyden ryhmätyön kannalta.

Suominen oppi projektin aikana ryhmädynamiikasta ja siitä, kuinka
tärkeää tiimin on kokoontua yhdessä johonkin tilaan työstämään
projektia.

Liukkonen oppi testaamisen tärkeydestä ryhmässä toteutettavan projektin
onnistumisen kannalta. (Ja kirjoittamaan itsestään kolmannessa persoonassa.)

Toisaalta vaikka kaikki ei ollut helppoa, lopputuloksena syntyi kuitenkin
toimiva sovellus (LOL). Projekti oli muutenkin hyvin opettavainen kokemus.

\end{document}
